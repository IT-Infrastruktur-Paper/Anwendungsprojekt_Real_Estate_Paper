The programming JavaScript is without a doubt the most commonly used programming language on the planet, in fact, about 63\% of developers are using JavaScript according to a 2023 study from Stackoverflow (!Quelle Stackoverflow Survey 2023). This marks JavaScipts 11th year in a row as the most popular programming language. Because of its popularity and importance for this research it is necessary to look further into the history of JavaScript and why it so important for web development.

The programming language JavaScript was initially invented in the year 1995 by Brendan Eich(w3). Two years later it was presented to the ECMA (European Computer Manufacturers Association) to become the ECMA-262 standard in the year 1997 (w3). After the first version (E1), which was solely supported by IE4 (Internet Explorer).
After a lot of iterations, discussions and new major releases, the most important version in the history of JavaScript is version number 5 (E5), which was the first version to be supported by all major browser at the time.

It is also important to declare the strict difference between the two programming languages JavaScript and Java since they have name similarities. Although they where developed in the same year (1995) they are different in its core. The main difference is that JavaScript is a lightweight, dynamically interpreted scripting-language. Java on the other hand is an object-oriented programming language which is made to be more robust and platform independent, leading with the promise "Write Once, Run Anywhere".

In order grasp the complexity and importance of JavaScript, it is necessary to get a basic understanding of it. Therefore ChatGPT has been asked to summarize the available resources on JavaScript with the following result:

"JavaScript is a high-level, dynamic, interpreted programming language primarily utilized for client-side web development. It employs a prototype-based object-oriented paradigm, featuring first-class functions and asynchronous programming capabilities. JavaScript is commonly embedded within HTML documents and executed by web browsers to manipulate the Document Object Model (DOM) dynamically, enabling interactive user interfaces and dynamic content generation. Its versatility extends beyond web development, as it can also be employed for server-side scripting, mobile application development, and desktop application development through frameworks like Node.js. JavaScript's syntax is influenced by C and Java, offering a rich ecosystem of libraries, frameworks, and tools for application development. It serves as a fundamental technology underpinning modern web applications, facilitating the creation of dynamic and interactive experiences across various platforms." (!Quelle/Prompt einfügen!)

The main takeaway from ChatGPT's summary is that the programming language JavaScript is of utmost importance for the development and usage of modern websites. Because without JavaScript, many parts of a website would not be dynamic as we have gotten used to it by now but rather static in its code and interaction (!Quelle as cited in freecodecamp). After all, without JavaScript "all you would have on the web would be HTML and CSS" (!Quelle freecodecamp) and commonly known website design-clues like a full-page drop down menu or content that is dynamically loaded into the websites body would be missing without the existence of JavaScript.

After those insights it is further necessary to take a look at common problems and setbacks of JavaScript. One common problem with JavaScript is its ...