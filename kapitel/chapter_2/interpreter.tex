Ein Interpreter, im Kontext von Programmierung, ist eine Software, die Quellcode Zeile für Zeile ausliest und den Quellcode zu diesem Zeitpunkt, der sogenannten Laufzeit, direkt ausführt. Der Interpreter analysiert den Quellcode also während der Ausführung und erkennt potenzielle Fehler ebenfalls erst während der Ausführung. Ein Beispiel für eine Programmiersprache, welche auf einen Interpreter zurückgreift, ist Java. (Vgl. Was ist ein Interpreter?, 2019, S. 1).
Die Funktionsweise eines Interpreters ermöglicht es also einem Programm den Quellcode direkt auszuführen, ohne ihn zuvor vollständig übersetzen zu müssen. Obwohl eine Übersetzung stattfindet, geschieht dies im Gegensatz zum Compiler nicht getrennt von der Ausführung. Für jede Zeile im Quellcode, erfolgt eine sofortige Aktion in Form der Ausführung des zugrunde liegenden Codes. Die Reihenfolge dieser Aktionen wird durch die Anweisungen im Quellcode festgelegt (Vgl. Was ist ein Interpreter?, 2019, S. 1).
Der Vorteil eines Interpreters liegt darin, Fehler die beim Programmieren entstanden sind, während der Ausführung sofort zu entdecken. Liegt ein Fehler im Quellcode vor, stoppt der Interpreter die weitere Ausführung des Programms und der Programmierer erkennt, dass genau an dieser Stelle ein Fehler im Quellcode vorliegen muss (Vgl. Was ist ein Interpreter?, 2019, S. 1).
Gleichzeitig bringt der Interpreter auch Nachteile mit sich. Der Interpreter arbeitet, im Vergleich zu einem Compiler, relativ langsam. Dies kommt daher, dass der Interpreter jede Zeile einzeln liest. Auch wenn sich Codezeilen wiederholen, werden diese vom Interpreter analysiert und erst dann erfolgreich ausgeführt (Vgl. Was ist ein Interpreter?, 2019, S. 1).
