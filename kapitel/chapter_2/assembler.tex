Assembler sind Programme, die Assemblercode in Maschinensprache, also Binärcode, umwandeln. Sie arbeiten direkt mit der jeweiligen Prozessorarchitektur des vorliegenden Systems und gelten als effizient und ressourcenschonend. Assembler übersetzen Code direkt in Binärcode, der manuell oder maschinell erstellt sein kann. Einige Compiler wandeln Programmcode aus höheren Programmiersprachen zunächst in Assemblercode um und verwenden dann einen Assembler, um den endgültigen Maschinencode zu generieren.
Assemblerprogramme können den gesamten Befehlssatz eines Prozessors nutzen, da jedem Assemblerbefehl ein entsprechender Befehl auf Maschinenebene entspricht. Im Gegensatz dazu, beschränken sich höhere Programmiersprachen auf eine Auswahl dieser Befehle und bieten daher teilweise die Möglichkeit, Assemblercode bei Bedarf in den Quellcode zu integrieren.
Jeder Prozessor hat eine eigene Architektur und einen eigenen Befehlssatz, was bedeutet, dass jeder Prozessor auch einen eigenen Assembler benötigt, um die Befehle entsprechend verarbeiten zu können. Nur in der spezifischen Assemblersprache, kann Code für einen bestimmten Prozessor verstanden und übersetzt werden. Die Assemblersprache verwendet mnemonische Kürzel für die internen Befehle des Prozessors, um logische und arithmetische Operationen, Registerzugriffe und die Steuerung des Programmflusses zu ermöglichen. Assemblerprogramme sind daher stark plattformabhängig und Programme müssen möglicherweise komplett neu entwickelt werden, um auf eine andere Architektur oder Plattform übertragen werden zu können. Ein Beispiel hierfür sind Spiele, welche sowohl für den Desktop-PC mit Intel Prozessor als auch für die Playstation 5 mit AMD Prozessor entwickelt werden.
Auf Grund der Nähe zur Prozessorarchitektur, bietet Assemblercode den Vorteil schnell in Binärcode, also Maschinencode, umgewandelt werden zu können. 
Heutzutage wird reine Assemblerprogrammierung selten verwendet, da sowohl Systemspeicher in verschiedenen Formen als auch grundsätzliche Rechenleistung, kostengünstig zu haben sind und der Aufwand hauptsächlich für die Optimierung zeitkritischer Systeme oder zu Lehrzwecken betrieben wird. Assemblersprache wirkt im Vergleich zu Hochsprachen oft limitiert und umständlich, da komplexe Operationen nicht Teil des Befehlssatzes sind und längere Programme erforderlich sind. Sie sind auch schwer zu verwalten, da der minimalistische Code schwer nachvollziehbar ist und eine sehr genaue Dokumentation im Quellcode erfordert.
