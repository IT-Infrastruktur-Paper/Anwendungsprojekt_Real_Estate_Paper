\newpage
\section{Discussion} \label{discussion}
The following chapter will discuss the previously described test results and argument whether or not they confirm or refuse the in chapter 3.2.1 defined null hypothesis. 

In general, however, it can be said in advance that the results of this paper clearly show that WASM offers clear performance advantages in certain scenarios. In particular, when executing computationally intensive algorithms, WASM was able to demonstrate a significant improvement in CPU utilization and overall performance compared to JavaScript. This observation confirms the perception of WASM as an advanced technology that enables more efficient execution of code on web browsers. 
A key aspect of our discussion is to analyze the underlying mechanisms that drive these performance differences. By examining CPU utilization patterns, memory consumption, and DOM manipulation efficiency, we were able to identify the specific strengths and weaknesses of WASM compared to JavaScript. These findings provide a deep insight into the technical aspects of the two platforms and help to make informed decisions when choosing the appropriate technology for specific use cases.

\textbf{Hypothesis Test 1} \newline
The null hypothesis for the first comparison of JavaScript and WASM based on the CPU performance was \dq WASM Runtime is faster than JavaScript Runtime\dq . Based on the results of the first test it can be said that WASM is about 3.27 times faster than JavaScript in executing the algorithm. Therefore the null hypothesis is confirmed and the alternate hypothesis is rejected. This result comes from the underlying technological advantage of WASM being low-level on the one hand and being compiled instead of interpreted on the other hand\footcite{bigelow_was_nodate}.

\textbf{Hypothesis Test 2} \newline
The null hypothesis for the second test comparing JavaScript and WASM based on the speed of RAM read and write operations was \dq WASM Runtime is faster than JavaScript Runtime\dq . Based on the results of the second test it can be said that WASM is about 4.67 times faster than JavaScript in executing the read and write operations. Therefore the null hypothesis is confirmed and the alternate hypothesis is rejected. As with the previous result, it again boils down to the underlying technological advantage of WASM being low-level on the one hand and being compiled instead of interpreted on the other hand\footcite{bigelow_was_nodate}.

\textbf{Hypothesis Test 3} \newline
The null hypothesis for the last test comparing JavaScript and WASM based on the efficiency in which they executed DOM manipulation commands was \dq JavaScript Runtime is faster than WASM Runtime\dq . Based on the results of the last test it can be said that, again, WASM is about 1.13 times faster than JavaScript in manipulating the DOM by adding div elements. In contrast to the tests before, this time the null hypothesis is rejected and the alternate hypothesis, which is \dq WASM Runtime is faster than JavaScript Runtime\dq , is confirmed. Since our null hypothesis was rejected, further explanation is necessary. Initially, the null hypothesis was based on the difficulty that WASM, regardless of its computational advantage, needs to 'pass' its result through JavaScript to the DOM in order to manipulate it. Therefore it is rather surprising that the results show that WASM is still faster than JavaScript in this discipline.