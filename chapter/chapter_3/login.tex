Eine Seite, bzw. Komponente ist die Login Seite, welche unter „rfinnotrade/frontend/src/components/pages/Login.jsx“ abgelegt ist.
Die Login.jsx Datei definiert die Login Seite unseres Frontends. Sie hat grundsätzlich die Funktion, den Benutzern die Anmeldung per E-Mail oder Benutzernamen mit dem dazugehörigen Passwort zu ermöglichen.
HIER CODE EINFÜGEN, AB ZEILE 32 bis 73
Das User-Interface (UI) ist ebenfalls hier implentiert und besteht aus einem Eingabefeld für die E-Mail und den Benutzernamen, einem Eingabefeld für das Passwort und einer Schaltfläche zum Absenden der Login-Daten. Wenn der Login fehlschlät, erscheint eine Fehlermeldung unterhalb des Formulars.
Das Styling der Seite wird über eine ausgelagerte Login.css Datei gesteuert.
Um den Login eines Benutzers durchzuführen, werden in der Login Komponente verschiedene Informationen verwaltet
HIER CODE EINFÜGEN, AB ZEILE 9 bis 11 
Wie hier zu sehen ist, verwaltet die Login Komponente password, identifiert und error, um die Logindaten oder Fehler bei der Anmeldung innerhalb der Komponente zu speichern und verarbeiten zu können.
Sendet der Benutzer den Login per Schaltfläche ab, wird die Funktion „handleSubmit“ in Zeile 41 aufgerufen. Diese sendet einen POST-Request an die Login-API von Strapi, überprüft dort die Anmeldedaten und der Benutzer wird bei erfolg an die Admin-Seite der Webseite weitergeleitet.
HIER CODE EINFÜGEN, AB ZEILE 17 bis 30
Außerdem wird die Funktion „onLoginSuccess“ ausgeführt, damit der Authentifizierungsstatus des Benutzers aktualisiert wird.