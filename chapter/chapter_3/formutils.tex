FormUtils.jsx ist ein Hilfsmodul, das verschiedene nützliche Funktionen für den Umgang mit Formularen und den daraus resultierenden Daten bereitstellt. Sie ist unter „rfinnotrade/frontend/src/utils/formUtils.jsx“ zu finden. Die Datei unterstützt bei der Initialisierung und Bearbeitung von Eingabefeldern.
Zwei wichtige implementierte Funktionen sind fetchContentType und fetchComponent.

\begin{lstlisting}[language=JavaScript, caption={formutils.jsx fetch-Funktionen}, label={lst:formutilsjsxFetchFunktionen}]
export const fetchContentType = async (contentType) => {
    try {
        const response = await axios.get(
        `${import.meta.env.VITE_STRAPI_BASE_URL}api/content-type-builder/content-types/${contentType}?populate=*`,
        { withCredentials: true }
        );
        return response.data.data.schema.attributes;
    } catch (err) {
        console.error(`Error fetching content type ${contentType}:`, err);
        return null;
    }
    };
    
    export const fetchComponent = async (componentName) => {
    try {
        const response = await axios.get(
        `${import.meta.env.VITE_STRAPI_BASE_URL}api/content-type-builder/components/${componentName}?populate=*`,
        { withCredentials: true }
        );
        return response.data.data.schema.attributes;
    } catch (err) {
        console.error(`Error fetching component ${componentName}:`, err);
        return null;
    }
    };
\end{lstlisting}

Sie sind darauf ausgelegt, die Datenstrukturen für Formulare im Backend per API abzurufen, und mithilfe der dort hinterlegten Strukturen dynamische Formulare zu ermöglichen. Die dafür genutzte API ist die Content-Type-Builder API von Strapi.
Anknüpfend an diese beiden Funktionen, gibt es die Funktion renderField.

\begin{lstlisting}[language=JavaScript, caption={formutils.jsx renderField-Funktion}, label={lst:formutilsjsxRenderFieldFunktion}]
export const renderField = (key, field, parentPath = '', newPage, setNewPage, setError, isEditing = false) => {
    const fieldPath = parentPath ? `${parentPath}.${key}` : key;
    
    if (field.type === 'component') {
        return (
        <div key={fieldPath} className="component-group">
            <h4>{field.displayName || key}</h4>
            {Object.entries(field.fields).map(([subKey, subField]) => 
            renderField(subKey, subField, fieldPath, newPage, setNewPage, setError, isEditing)
            )}
        </div>
        );
    }
\end{lstlisting}

Diese Funktion nutzt die geladenen Strukturen und erstellt dann tatsächlich und basierend auf den geladenen Strukturen die Eingabefelder im Formular. RenderFields wird zum Beispiel in BrandItem.jsx verwendet.
Außerdem wird in dieser Datei der Upload von Dateien über ein Formular ermöglicht.

\begin{lstlisting}[language=JavaScript, caption={formutils.jsx Datei-Upload-Funktion}, label={lst:formutilsjsxDateiUpload}]
export const handleFileChange = async (e, fieldName, setNewPage, setError) => {
    const file = e.target.files[0];
    if (!file) return;
    
    try {
        const formData = new FormData();
        formData.append('files', file);
    
        const uploadResponse = await axios.post(
        `${import.meta.env.VITE_STRAPI_BASE_URL}api/upload`,
        formData,
        {
            withCredentials: true,
            headers: {
            'Content-Type': 'multipart/form-data',
            },
        }
        );
\end{lstlisting}

Auch hier erfolgt wieder die Kommunikation mit Strapi. Dieses Mal über die Strapi-Upload-API.
Ebenso wird hier eine Funktion zur Initialisierung von vollständig neuen Seiten, bzw. Brands realisiert.

\begin{lstlisting}[language=JavaScript, caption={formutils.jsx Neue Seite initialisieren}, label={lst:formutilsjsxNeueSeite}]
export const initializeNewPage = (fields) => {
    const initialPage = {};
    for (const [key, field] of Object.entries(fields)) {
        if (field.required) {
        if (field.type === 'component') {
            initialPage[key] = initializeNewPage(field.fields);
        } else {
            switch (field.type) {
            case 'boolean':
                initialPage[key] = field.default || false;
                break;
            case 'media':
                initialPage[key] = null;
                break;
            case 'text':
            case 'string':
            case 'uid':
                if (field.regex === '^#?([A-Fa-f0-9]{6}|[A-Fa-f0-9]{3})$') {
                initialPage[key] = field.default || '#f7f7f7';
                } else {
                initialPage[key] = '';
                }
                break;
            default:
                initialPage[key] = null;
            }
        }
        }
    }
    return initialPage;
    };
\end{lstlisting}

Mithilfe dieser Funktion wird ein neues Objekt für eine Seite erstellt, das auf den Felddefinitionen aus der Strapi-API basiert. Diese Felder sind erstmal inhaltslos und werden später durch die Eingaben des Users mit Inhalt gefüllt.