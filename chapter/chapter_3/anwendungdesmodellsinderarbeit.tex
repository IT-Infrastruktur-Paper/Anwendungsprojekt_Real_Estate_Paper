\textbf{Business Understanding}

\textbf{Data Understanding}

\textbf{Data Preparation}

\textbf{Modeling}

\textbf{Evaluation}

\textbf{Deployment}





Business Understanding

    Zieldefinition: Bewertung, ob der Deutschlandatlas als Datenquelle geeignet ist, um Immobilienkaufpreise in Deutschland zu ermitteln und vorherzusagen.

    Ableitung konkreter Analyseziele: Entwicklung eines Vorhersagemodells für Immobilienpreise auf Basis der im Deutschlandatlas verfügbaren Indikatoren.

    Identifikation relevanter Stakeholder (z.B. Immobilienunternehmen, Investoren, Politik).

Data Understanding

    Sichtung und Analyse der im Deutschlandatlas bereitgestellten Daten (z.B. sozioökonomische Indikatoren, regionale Kennzahlen).

    Überprüfung der Datenverfügbarkeit bezüglich Immobilienkaufpreisen und potenziell relevanter Einflussfaktoren.

    Erste Exploration: Identifizieren von Korrelationen zwischen Atlasdaten und Immobilienpreisen.

    Erkennen von Datenlücken oder potenziellen Verzerrungen.

Data Preparation

    Auswahl relevanter Variablen aus dem Deutschlandatlas (z.B. Einkommen, Arbeitslosenquote, Infrastruktur).

    Ergänzung durch externe Quellen, falls direkte Immobilienpreisdaten fehlen.

    Datenbereinigung: Umgang mit fehlenden Werten, Ausreißern und Inkonsistenzen.

    Transformation und Aggregation der Daten auf die gewünschte regionale Ebene (z.B. Landkreise, Städte).

Modeling

    Auswahl geeigneter Modellierungsmethoden (z.B. lineare Regression, Random Forest).

    Training der Modelle mit den aufbereiteten Daten.

    Feature Engineering: Auswahl und Kombination von Variablen, die Immobilienpreise beeinflussen könnten.

    Durchführung von Modellvalidierungen (z.B. Kreuzvalidierung).

Evaluation

    Bewertung der Modellgüte anhand geeigneter Metriken (z.B. RMSE, R²).

    Überprüfung, ob die Modelle auf Basis der Deutschlandatlas-Daten verlässliche Vorhersagen ermöglichen.

    Analyse der wichtigsten Einflussfaktoren im Modell.

    Reflexion, inwieweit die Forschungsfrage beantwortet werden kann.

Deployment

    Präsentation der Ergebnisse in verständlicher Form (z.B. Visualisierungen, Karten).

    Diskussion der Implikationen für Praxis und Forschung.

    Empfehlungen für die Nutzung des Deutschlandatlas in der Immobilienmarktanalyse.

    Hinweise auf Limitationen und mögliche Weiterentwicklungen.
