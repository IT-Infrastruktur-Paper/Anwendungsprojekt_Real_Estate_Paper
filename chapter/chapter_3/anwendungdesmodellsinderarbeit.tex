\textbf{Business Understanding}
In der ersten Phase hat sich das Projektteam folgendes Ziel in Form einer Forschungsfrage gesetzt: "Ist der Deutschlandatlas als Datenquelle geeignet, um Immobilienkaufpreise in Deutschland zu ermitteln und vorherzusagen?"
Das Ziel dieser Arbeit ist es, ein KI-gestütztes Vorhersagemodell für Immobilienpreise zu entwickeln, das auf den im Deutschlandatlas verfügbaren Indikatoren basiert. Das aus dieser Arbeit resultierende Modell ist zugeschnitten auf relevante Stakeholder wie Immobilienunternehmen oder -investoren.

\textbf{Data Understanding}
In der Phase des Data Understanding hat das Projektteam die im Deutschlandatlas verfügbaren Daten gesichtet. Dabei wurde festgestellt, dass der Deutschlandatlas verschiedenste sozioökonomischen Faktoren und regionalen Kennzahlen bereitstellt, die potenziell für die Modellierung eines Vorhersagemodells für Immobilienpreise relevant sind. Des Weiteren wurde geprüft ob die im Deutschland bereitgestellten Daten genügen oder ob noch weitere externe Datenquellen hinzugezogen werden müssen. Erste explorative Analysen haben eine erkennbare Verbindung zwischen den Atlasdaten und Immobilienpreisen aufgezeigt, jedoch genau so Datenlücken und mögliche Verzerrungen identifiziert.

\textbf{Data Preparation}
In der nächsten Phase wurden die Daten des Deutschlandatlas aufbereitet und durch drei externe Datenquellen erweitert:
\begin{itemize}
    \item Verwaltungsgebiete vom Bundesamt für Kartographie und Geodäsie
    \item Übersetzungsaten für ARS und AGS zu Postleitzahlen vom Statistischen Bundesamt
    \item Immobilienpreisdaten von Engel & Völkers
\end{itemize}


\textbf{Modeling}
Beim modellieren der Daten und erstellen des Vorhersagemodells wurde ein zweistufiger Ansatz verfolgt. Zunächst wurde ein Regressionsmodell entwickelt, das die Immobilienpreise auf Basis der Deutschlandatlas-Daten vorhersagt. Anschließend wurde ein Klassifikationsmodell erstellt, das die Immobilien in verschiedene Preiskategorien einteilt. Beide Modelle wurden mit den aufbereiteten Daten trainiert und validiert.

\textbf{Evaluation}

\textbf{Deployment}



Data Preparation

    Datenbereinigung: Umgang mit fehlenden Werten, Ausreißern und Inkonsistenzen.

    Transformation und Aggregation der Daten auf die gewünschte regionale Ebene (z.B. Landkreise, Städte).

Modeling

    Auswahl geeigneter Modellierungsmethoden (z.B. lineare Regression, Random Forest).

    Training der Modelle mit den aufbereiteten Daten.

    Feature Engineering: Auswahl und Kombination von Variablen, die Immobilienpreise beeinflussen könnten.

    Durchführung von Modellvalidierungen (z.B. Kreuzvalidierung).

Evaluation

    Bewertung der Modellgüte anhand geeigneter Metriken (z.B. RMSE, R²).

    Überprüfung, ob die Modelle auf Basis der Deutschlandatlas-Daten verlässliche Vorhersagen ermöglichen.

    Analyse der wichtigsten Einflussfaktoren im Modell.

    Reflexion, inwieweit die Forschungsfrage beantwortet werden kann.

Deployment

    Präsentation der Ergebnisse in verständlicher Form (z.B. Visualisierungen, Karten).

    Diskussion der Implikationen für Praxis und Forschung.

    Empfehlungen für die Nutzung des Deutschlandatlas in der Immobilienmarktanalyse.

    Hinweise auf Limitationen und mögliche Weiterentwicklungen.
