Als nächstes gehen wir auf das Admin-Center ein, welches unter „rfinnotrade/frontend/src/components/pages/admincenter.jsx“ gespeichert ist. Die Admin-Center Komponente stellt das Hauptverwaltungscenter für die Brands dar. Hier können neue Seiten erstellt, Seiteninhalte bearbeitet und laufende Docker-Container kontrolliert werden.
Auch diese Komponente nutzt wieder verschiedene States.

\begin{lstlisting}[language=JavaScript, caption={AdminCenter.jsx states}, label={lst:admincenterjsxStates}]
function AdminCenter() {
    const [brands, setBrands] = useState([]);
    const [error, setError] = useState('');
    const [isModalOpen, setIsModalOpen] = useState(false);
    const [newPage, setNewPage] = useState({});
    const [activeContainers, setActiveContainers] = useState([]);
    const [pageFields, setPageFields] = useState({});
\end{lstlisting}

Brands speichert die Liste aller erstellten Brands. Der State isModalOpen steuert die Sichtbarkeit des Modals zur Erstellung von neuen Seiten und newPage speichert die Daten der neuen Seite die erstellt werden soll ab. ActiveContainers enthält die Informationen zu aktuell laufenden Docker-Containern und pageFields speichert die Felder des content-types für Seiten.
Innerhalb der Komponente sind außerdem verschiedenste Funktionen implementiert. Im Folgenden gehen wir auf drei wichtige Funktionen ein.

\begin{lstlisting}[language=JavaScript, caption={fetchBrands-Funktion}, label={lst:admincenterjsxFetchBrandsFunktion}]
const fetchBrands = async () => {
try {
    const response = await axios.get(
    `${import.meta.env.VITE_STRAPI_BASE_URL}api/pages?populate=ColorPalette`, 
    { withCredentials: true }
    );
    setBrands(response.data.data);
} catch (err) {
    setError('Failed to fetch brands');
    console.error('Error fetching brands:', err);
}
};
\end{lstlisting}

Diese Funktion ruft die Liste der Brands über eine API ab und speichert diese dann im state brands. Nach dieser Vorgehensweise funktioniert auch fetchActiveContainers, zum abrufen und speichern der laufenden Container.

\begin{lstlisting}[language=JavaScript, caption={fetchActiveContainers-Funktion}, label={lst:admincenterjsxFetchActiveContainersFunktion}]
const fetchActiveContainers = async () => {
    try {
        const response = await axios.get(
        `${import.meta.env.VITE_STRAPI_BASE_URL}api/docker/list`,
        { withCredentials: true }
        );
        setActiveContainers(response.data);
    } catch (err) {
        setError('Failed to fetch running containers');
        console.error('Error fetching running containers:', err);
    }
    };
\end{lstlisting}

In Zeile 4 ist ein beispielhafter API-Endpunkt zu sehen. Das Frontend greift über diese API auf das Backend in Strapi zu und lädt von dort aus die laufenden Docker-Container. 

\begin{lstlisting}[language=JavaScript, caption={fetchPageFields-Funktion}, label={lst:admincenterjsxFetchPageFieldsFunktion}]
const fetchPageFields = async () => {
    try {
        // Fetch the main content type
        const mainFields = await fetchContentType('api::page.page');
        if (!mainFields) return;

        // Process components recursively
        const processedFields = {};
        for (const [key, field] of Object.entries(mainFields)) {
        if (field.type === 'component') {
            // Fetch and process component fields
            const componentFields = await fetchComponent(field.component);
            processedFields[key] = {
            ...field,
            fields: componentFields
            };
        } else {
            processedFields[key] = field;
        }
        }
\end{lstlisting}

Diese Funktion sorgt dafür, dass Felddefinitionen des Seiten-Content-Types sowie zugehörige Komponenten geladen und die Struktur der neuen Seite initialisiert werden. Dies ist also die Hauptfunktion, um die in den Feldern vom Benutzer eingetragenen Daten für die neue Seite zu übernehmen und die Initialisierung der neuen Brand zu starten.
Auch im admin-center ist wieder der eigentliche Webseitenaufbau in der html-nahen jsx-Syntax zu finden.

\begin{lstlisting}[language=JavaScript, caption={AdminCenter.jsx Aufbau}, label={lst:admincenterjsxAufbau}]
return (
    <div className="admin-center">
        <div className="brand-management">
        <div className="subheader">
            <h2 className="center-title" > Admin Center </h2>
            <button onClick={openModal} className="add-brand-btn">+ New Page</button>
        </div>
        
        {isModalOpen && (
            <div className="modal-overlay">
            <div className="modal-content">
                <h3>Neue Seite erstellen</h3>
                <form onSubmit={handleSubmit}>
                {Object.entries(pageFields).map(([key, field]) => 
                    renderField(key, field, '', newPage, setNewPage, setError)
                )}
                <div className="modal-actions">
                    <button type="button" onClick={closeModal}>Cancel</button>
                    <button type="submit">Seite erstellen</button>
                </div>
                </form>
            </div>
            </div>
        )}

        <div className="brand-list">
            {brands.map((brand, index) => {
            const isRunning = activeContainers.some(container => container.name === `subbrand-${brand.documentId}`);
            return (
                <BrandItem
                key={brand.id || index}
                name={brand.BrandName || `Brand ${index + 1}`}
                page={brand}
                refreshBrands={() => fetchBrands()}
                isRunning={isRunning}
                fetchActiveContainers={fetchActiveContainers}
                pageFields={pageFields}
                isEditing={activeEditBrandId === brand.documentId} // Check if this brand is being edited
                toggleEdit={() => toggleEditBrand(brand.documentId)} // Toggle edit mode
                />
            );
            })}
        </div>
        </div>
    </div>
    );
}
\end{lstlisting}

Der Aufbau der Webseite ist dabei grundsätzlich in zwei Teile unterteilt. Die eigentliche admin-center Seite und die darin integrierte brand-list.