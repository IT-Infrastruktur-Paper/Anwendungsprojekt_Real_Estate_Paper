Diese Datei, abgelegt unter „rfinnotrade/subbrand/src/App.jsx“ ist die zentrale Komponente des Subbrand Bereichs. Sie dient als Einstiegspunkt für React, definiert das Grundgerüst und steuert das Laden der dynamischen Inhalte. Sie ist also zum Beispiel für das Laden von Farben, Schriftarten oder Beschreibungen der einzelnen Brands verwantwortlich.

\begin{lstlisting}[language=JavaScript, caption={API.jsx}, label={lst:apijsx}]
const STRAPI_BASE_URL = import.meta.env.VITE_STRAPI_BASE_URL;
const documentId = import.meta.env.VITE_DOCUMENTID

export const fetchPageByContentId = async () => {
    try {
    const response = await fetch(`${STRAPI_BASE_URL}/api/pages/${documentId}?populate=*`);
    const result = await response.json();
    console.log(result)
    return result.data; // Assuming contentId is unique
    } catch (error) {
    console.error('Error fetching page data:', error);
    throw error;
    }
};
\end{lstlisting}

Die Funktion fetchPageByContentID ist aus der Datei API.jsx im selben Ordner importiert und dient zum Laden der gewünschten Seiteninhalte per API.

\begin{lstlisting}[language=JavaScript, caption={App.jsx state}, label={lst:appjsxState}]
try {
    const data = await fetchPageByContentId();
    setPageData(data);
\end{lstlisting}

Im Anschluss daran werden die Daten der Seite in den state pageData geladen.

\begin{lstlisting}[language=JavaScript, caption={App.jsx Imports}, label={lst:appjsxImports}]
import Header from './components/HeaderBrands';
import ShopifyCollection from './components/ShopifyCollection';
import AboutUs from './components/AboutUs';
import Footer from './components/FooterBrands';
import Home from './pages/Home';    
\end{lstlisting}

Außerdem werden die einzelnen Komponenten der Seite ebenfalls hier importiert. Sie sind der eigentliche Bestandteil der Subbrands und werden hier wieder vereint und strukturiert. Wenn man sich nun den gesamten useEffect-Hook anschaut, kann man darüber hinaus die dynamische Integration der CSS-Inhalte sehen.

\begin{lstlisting}[language=JavaScript, caption={App.jsx useEffect-Funktion}, label={lst:appSubbrandsUseEffectFunktion}]
useEffect(() => {
    const loadPage = async () => {
        try {
        const data = await fetchPageByContentId();
        setPageData(data);
        // Dynamically set CSS variables
        if (data?.ColorPalette) {
            const { primarycolor, secondarycolor, tertiarycolor } = data.ColorPalette;
            document.documentElement.style.setProperty('--primary-color', primarycolor);
            document.documentElement.style.setProperty('--secondary-color', secondarycolor);
            document.documentElement.style.setProperty('--tertiary-color', tertiarycolor);
        }
        document.documentElement.style.setProperty('--font', data?.Font || '');
        document.documentElement.style.setProperty('--font-url', data?.FontURL || '');

        setLoading(false);
        } catch (error) {
        console.error("Failed to fetch page", error);
        setLoading(false);
        }
    };
\end{lstlisting}

Hier zu sehen in Zeile 9 bis 11. Dies ermöglicht, dass jede Subbrand ihre eigenen Gestaltungsmöglichkeiten erhält.
Damit Inhalte der Seite erst dann angezeigt werden, wenn alle Daten vollständig geladen sind, gibt es den loading state und dessen setLoading Funktion.

\begin{lstlisting}[language=JavaScript, caption={App.jsx setLoading}, label={lst:appjsxSetLoading}]
setLoading(false);
} catch (error) {
    console.error("Failed to fetch page", error);
    setLoading(false);
}
};
\end{lstlisting}

Mithilfe dieses states und der zugehörigen Funktion, kann der Ladestatus jederzeit abgefragt und der Status, wie im Beispiel zu sehen ist, neu gesetzt werden.
Der Router in der Datei verwaltet wiederum, wie schon zuvor in den anderen Bereichen unseres Projekts, die Navigation zwischen den verschiedenen Bereichen einer Subbrand und wie diese erreichbar sind.

\begin{lstlisting}[language=JavaScript, caption={App.jsx Router}, label={lst:appjsxRouter}]
<Router>
<div className="App">
    <Header />
    <Routes>
    <Route path="/" element={<Home pageData={pageData} />} />
    <Route path="/about" element={<AboutUs />} />
    </Routes>
    <Footer />
</div>
</Router>
\end{lstlisting}