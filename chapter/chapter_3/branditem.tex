Der Code in dieser Datei, welche unter „rfinnotrade/frontend/src/components/BrandItems.jsx“ abgelegt ist, definiert eine weitere React-Komponente, die für die Anzeige und Verwaltung einzelner Brand- oder Seiteninformationen genutzt wird.
In der Komponente werden Funktionen wie das Starten und Stoppen von Containern, das Bearbeiten von Branddaten und das Löschen einer Brand bereitgestellt. Die Datei bietet also wichtige Verwaltungsfunktionen für den Nutzer.
Eine sehr nützliche Funktion im Admin Center bietet die Editierfunktion für die einzelnen Brands, welche über den Edit Button aufgerufen werden kann.
Zu Beginn des Codes wird editedPage als state definiert.

\begin{lstlisting}[language=JavaScript, caption={branditem.jsx}, label={lst:branditemjsx}]
const [editedPage, setEditedPage] = useState(page);
\end{lstlisting}

Über diesen state werden Änderungen an den Daten einer Brand, wie zum Beispiel der Name oder ein Beschreibungstext, gespeichert. 
Sollten tatsächlich Veränderungen an einer Seite vorgenommen worden sein, wird beim Speichern der geänderte Inhalt mit dem vorherigen Inhalt verglichen und nur die veränderten Daten per API-PUT Anfrage ans Backend gesendet.

\begin{lstlisting}[language=JavaScript, caption={branditem.jsx Editierung}, label={lst:branditemjsxEditierung}]
if (Object.keys(updateData).length > 0) {
    await axios.put(
        `${import.meta.env.VITE_STRAPI_BASE_URL}api/pages/${page.documentId}`,
        { data: updateData },
        { withCredentials: true }
    );
    toggleEdit();
    refreshBrands();
}
\end{lstlisting}

Im Anschluss daran wird der editedPage Status über toggleEdit verändert und über refreshBrands wird lediglich die fetchBrands Funktion, die wiederum im admin-center definiert ist, ausgeführt und somit die Brandlist aktualisiert.
Eine wichtige Funktion im Code übernimmt isEditing. IsEditing steuert, ob sich die Brand-Komponente im Bearbeitungsmodus befinet oder nicht. Über diese Funktion kann also der Status des Containers abgefragt werden.
Passend dazu, kann der Container über zwei Button gestartet oder gestoppt werden.

\begin{lstlisting}[language=JavaScript, caption={branditem.jsx Start/Stop Container-Funktionen}, label={lst:branditemjsxStartStopContainer}]
const handleStartContainer = async () => {
    try {
        setIsLoading(true);
        
        // If container info is empty or status is empty, create container first
        if (!containerInfo || !containerInfo.status) {
        await handleCreateContainer();
        }

        const response = await axios.post(
        `${import.meta.env.VITE_STRAPI_BASE_URL}api/docker/start`,
        { 
            documentId: page.documentId,
            brandName: page.BrandName
        },
        { withCredentials: true }
        );
        if (response.data.success) {
        alert(
            `Container started successfully!\n` +
            `Name: ${response.data.containerName}\n` +
            `Port: ${response.data.port}`
        );
        fetchActiveContainers();
        }
    } catch (err) {
        setError('Failed to start container');
        console.error('Error starting container:', err);
    } finally {
        setIsLoading(false);
    }
    };

    const handleStopContainer = async () => {
    try {
        setIsLoading(true);
        const response = await axios.post(
        `${import.meta.env.VITE_STRAPI_BASE_URL}api/docker/stop`,
        { 
            documentId: page.documentId,
            brandName: page.BrandName
        },
        { withCredentials: true }
        );
        if (response.data.success) {
        alert(`Container stopped successfully!`);
        fetchActiveContainers();
        }
    } catch (err) {
        setError('Failed to stop container');
        console.error('Error stopping container:', err);
    } finally {
        setIsLoading(false);
    }
    };
\end{lstlisting}

Hinter den beiden Buttons, sind diese beiden Funktionen hinterlegt. Wenn der Container läuft, wird nur der Stop Container Button angezeigt und wenn der Container nicht läuft, wird nur der Start Container Button angezeigt. Der Delete Button ist ähnlich implementiert, wobei in diesem Fall nicht nur der Container gestoppt, sondern die ganze Brandpage gelöscht wird. Dies bietet dem User die einfache und schnelle Möglichkeit, die Brandpage und dessen Containerinstanz zu verwalten.
Darüber hinaus ist auch eine visuelle Anzeige in die Brand List innerhalb des Admin Centers integriert. Hier wird über isRunning einfach der aktuelle Status abgefragt und für den Benutzer sichtbar gemacht.