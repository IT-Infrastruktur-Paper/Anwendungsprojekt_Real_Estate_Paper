Die nachfolgende Datei liegt unter „rfinnotrade/strapi/src/api/custom/controllers/custom.js“ und beinhaltet verschiedene Funktionen. Im Folgenden werde ich auf die Funktionen „login“, „logout“ und „checkAuthStatus“ eingehen, da diese im Kontext von Strapi und den bisher genannten Funktionen am relevantesten sind.

Als erstes gehen wir auf die Login-Funktion ein. Diese Funktion erledigt die Aufgabe, den eigentlichen Login bei einer Login-Anfrage von der Admin Seite zu verifizieren, Daten aus Strapi abzufragen und darüber hinaus alle notwendigen Daten in den Cookies zu setzen.

\begin{lstlisting}[language=JavaScript, caption={Login-Funktion}, label={lst:customjsLogin}]
async login(ctx) {
    const { body } = ctx.request;
    const hostname = "localhost";
    const absoluteURL = `http://${hostname}:${strapi.config.server.port}`;
    const sanitizeOutput = (user) => {
      const {
        password,
        resetPasswordToken,
        confirmationToken,
        ...sanitizedUser
      } = user; // be careful, you need to omit other private attributes yourself
      return sanitizedUser;
    };
    try {
      console.log("Tryin to login");
      // Now submit the credentials to Strapi's default login endpoint
      let { data } = await axios.post(`${absoluteURL}/api/auth/local`, body);
      const populatedUser = await strapi.entityService.findOne(
        "plugin::users-permissions.user",
        data.user.id,
        {
          populate: {
            role: {
              fields: ["type"],
            },
          },
        }
      );
      data.user = sanitizeOutput(populatedUser);
      // Set the secure cookie
      if (data && data.jwt) {
        ctx.cookies.set("jwt", data.jwt, {
          httpOnly: true,
          secure: false,
          SameSite: "None",
          maxAge: 1000 * 60 * 60 * 24 * 7, // 7 Day Age
        });
      }
      // Respond with the jwt + user data, but now this response also sets the JWT as a secure cookie
      return ctx.send(data);
    } catch (error) {
      console.log("An error occurred:", error.response);
      return ctx.badRequest(null, error);
    }
  },
\end{lstlisting}

Um diesen Vorgang zu realisieren, wird erneut der gesamte Webseitenkontext in Zeile 1 übergeben und daraus alle „request“-Daten in Zeile 2 geladen. Hier sind unter anderem die Eingaben des Benutzers gespeichert.
Von Zeile 5 bis Zeile 13, wird die Funktion „sanitizeOutput“ deklariert, welche erst später angewendet wird.
Ab Zeile 14 wird versucht mithilfe der übergebenen Login-Daten, eine Anfrage an Strapis eigentliche Login API durchzuführen. Ist diese Anfrage erfolgreich, werden die entsprechenden Benutzerdaten aus der Strapi Datenbank geladen, unter anderem die Benutzerrollen abgefragt.
Ist alles erledigt, wird eine Cookie mit allen notwendigen Informationen zurückgegeben und in den Webseitenkontext geladen.
Somit wurde die Login Abfrage erfolgreich über eine API durchgeführt und alle notwendigen Login Informationen anschließend in den Cookies der Webseite für weiter Abfragen gespeichert. Eine automatische Logout Funktion wurde in Zeile 68 ebenfalls großzügig implementiert. Diese würde spätestens nach 7 Tagen greifen, das Cookie auslaufen und der Logout somit automatisch erfolgen.

Der ebenfalls implementierten Logout-Funktion wird zu Beginn erneut der Webseitenkontext übergeben.

\begin{lstlisting}[language=JavaScript, caption={Logout-Funktion}, label={lst:customjsLogout}]
async logout(ctx) {
    try {
      // Clear the JWT cookie
      ctx.cookies.set('jwt', null, {
        httpOnly: true,
        secure: false,
        maxAge: 0,
      });
      
      return ctx.send({
        message: 'Successfully logged out',
      });
    } catch (error) {
      return ctx.badRequest(null, error);
    }
  },
\end{lstlisting}

In Zeile 4 wird dann zunächst das Cookie aus dem Kontext ausgelesen und und das JWT auf „NULL“ gesetzt, wodurch der Logout theoretisch bereits erfolgt ist. Zusätzlich wird der automatische Logout, welcher in der Login Funktion auf 7 Tage gesetzt wird, auf den Wert 0 gesetzt.
Dadurch löst sich das Cookie selber auf und der Logout würde ebenfalls erfolgen. Das Cookie wird am Ende wieder an den Webseitenkontext zurückgegeben.

Die Funktion checkAuthStauts hat die Aufgabe, den aktuellen Login Status bei Aufruf dieser Funktion abzufragen. Diese Funktion ergänzt die Login- und Logout-Funktion also um eine weiter wichtige Funktionalität.
Die Funktion ist ähnlich aufgebaut wie Login und Logout.

\begin{lstlisting}[language=JavaScript, caption={checkAuthStatus-Funktion}, label={lst:customjsCheckAuthStatus}]
async checkAuthStatus(ctx) {
    // Check for JWT in cookies first
    const jwtCookie = ctx.cookies.get('jwt');
    
    if (!jwtCookie) {
      return ctx.unauthorized('No JWT cookie found');
    }

    try {
      const decoded = verify(jwtCookie, strapi.config.get('plugin::users-permissions.jwtSecret'));
      return ctx.send({ isAuthenticated: true, user: decoded });
    } catch (err) {
      return ctx.unauthorized('Invalid token');
    }
  },
\end{lstlisting}

Diese Funktion überprüft zunächst, ob überhaupt ein JWT vorhanden ist. Sollte das nicht so sein, wird die Anfrage abgelehnt, da kein aktueller Login erfolgt ist. Sollte ein JWT vorhanden sein, wird dieses erneut verifiziert und der Benutzer kann, entsprechend seinen Berechtigungen, Anfragen durchführen.