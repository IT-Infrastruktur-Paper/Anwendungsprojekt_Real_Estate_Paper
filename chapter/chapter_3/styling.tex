In unserem Webprojekt haben wir uns für die klassische CSS-Variante entschieden. Für das Projekt bedeutet dies, dass wir die styling sheets ausgelagert haben und nur vereinzelt inline-styles verwendet haben.
Der Vorteil von ausgelagerten styling sheets ist, dass sie zentral verwaltet werden können und somit eine einheitliche Gestaltung des Projekts gewährleistet wird. Zudem können die styling sheets bei Bedarf einfach ausgetauscht werden, ohne dass der HTML-artige Code in React angepasst werden muss.
Die Verwendung von inline-styles ist in React ebenfalls möglich, jedoch wird davon abgeraten, da es die Lesbarkeit des Codes erschwert und die Wartbarkeit des Projekts erschwert. Zudem ist es nicht möglich, die inline-styles zentral zu verwalten, was zu einer inkonsistenten Gestaltung des Projekts führen kann.
Um möglichst einfach und dynamisch Inhalte wie Farben oder Fonts austauschen und wiederverwenden zu können, haben wir uns für die Verwendung von CSS-Variablen entschieden. Diese können zentral in einem der zuvor genannten styling sheets definiert und in den Komponenten der Webseite verwendet werden.

\begin{lstlisting}[language=JavaScript, caption={global.css}, label={lst:globalcss}]
:root{
    --main-color: #22593c;
    --bg-color: #d1d1d1;
    --acc-color: #f7f7f7;
    --hover-color: #409b6a;
    }
    
    @font-face {
    font-family: 'CustomFont';
    src: url('../../assets/fonts/BristoneThin.ttf') format('truetype'),
            url('../../assets/fonts/BristoneThin.woff') format('woff'),
            url('../../assets/fonts/BristoneThin.woff2') format('woff2');
    font-weight: normal;
    font-style: normal;
    }
\end{lstlisting}

In diesem Codeausschnitt sehen wir einen Teil des Codes aus einer global.css Datei. Zwischen Zeile 1 und 6 werden global für alle styling-sheets, innerhalb der gleichen React App, verfügbare Farben definiert. Diese können dann in den Komponenten der Webseite verwendet werden.
In Zeile 8 wird ein Font definiert, der in der gesamten Webseite verwendet werden kann. Dieser Font wird in der global.css Datei definiert und kann dann ebenfalls in den Komponenten der Webseite verwendet werden.
Zur Gestaltugn der Webseite haben wir zudem mit Flexbox gearbeitet. Flexbox ist ein Layout-Modell, das es ermöglicht, Elemente innerhalb eines Containers flexibel zu positionieren. Dies ermöglicht es, die Webseite auf verschiedenen Bildschirmgrößen und Geräten optimal darzustellen.

\begin{lstlisting}[language=JavaScript, caption={Flexbox}, label={lst:flexbox}]
    .header {
        position: sticky;
        top: 0;
        display: flex;
        flex-direction: row;
        justify-content: space-between;
        align-items: center;
        background-color: var(--main-color);
        height: 80px;
        width: 100%;
        color: var(--acc-color);
        z-index: 1000;
        padding: 0 20px;
    }
\end{lstlisting}

Hier haben wir einen header definiert, der im oberen Bereich der Webseite angezeigt wird. In Zeile 4 und 5 sieht man jeweils eine mögliche Implementation von Flexbox in CSS.