GitHub is a web-based version control system for software development that allows developers to manage their code in private or public repositories. Although there are many alternatives, such as GitLab, Bitbucket or SourceForge, we chose GitHub mainly because of its seamless integration with Microsoft Visual Studio Code, the chosen editor for this project.

The benefits of a centralised version control are obvious. On the one hand, it allows the entire project to be managed in one place, ensuring that it is always the latest version. On the other hand, a GitHub repository, even if set to private, facilitates collaborative and platform-independent work on the project.

In theory, GitHub's ability to create project and organisational structures could make it much easier to scale up research projects. In addition to managing such projects through Git pull and push requests, GitHub offers effective use of built-in features such as issues and discussions to challenge others' contributions, leading to the development of better and more refined ideas.

However, to maintain the focus and scope of this paper, we have chosen to use GitHub exclusively as a version control system. There is a repository called "WASMvsJS" for the development of the experiment (!Quelle GitHub Repo), and another called "ITInfraPaper" for the \LaTeX{} project (!Source GitHub Repo). Both repositories are publicly accessible and linked in the source directory.