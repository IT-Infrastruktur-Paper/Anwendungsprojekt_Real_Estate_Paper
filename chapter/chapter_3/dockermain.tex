Für Docker und zur Erstellung von Containern ist jeweils im Frontend, Backend und in Subbrands eine Dockerfile geschrieben, über die Docker Images erstellt werden. Alle einzelnen Komponenten laufen in einem eigens kreierten Container. Strapi im Backend, das admin Center im Frontend und auch die einzelnen Subbrands werden in einem eigenen Docker Container gestartet.
Die Container für Strapi und das Admin Center laufen jeweils automatisch und dauerhaft. Lediglich die Subbrands können über das Admin Center eigenhändig gestartet und gestoppt werden, wobei der initialie Start des Containers einer Subband immer manuell ausgeführt werden muss.

\begin{lstlisting}[language=JavaScript, caption={Middlewares.js}, label={lst:middlewaresjs}]
# Use the official Node.js image for building the React app
FROM node:alpine AS build

# Set the working directory for the build
WORKDIR /app

# Copy package.json and package-lock.json
COPY package*.json ./

# Install dependencies with increased memory limit
RUN node --max-old-space-size=2048 $(which npm) install

# Copy the rest of the application code
COPY . .
ARG VITE_DOCUMENTID
ENV VITE_DOCUMENTID=${VITE_DOCUMENTID}
ENV VITE_STRAPI_BASE_URL=http://www.rf-innotrade.de:1337
# Build the React app
RUN npm run build

# Use the official Nginx image for serving static files
FROM nginx:alpine

# Set the working directory in the container
WORKDIR /usr/share/nginx/html

# Remove the default Nginx static assets
RUN rm -rf ./*

# Copy the prebuilt React files from the build stage
COPY --from=build /app/dist /usr/share/nginx/html
COPY nginx.conf /etc/nginx/conf.d/default.conf

# Expose port 80
EXPOSE 80

# Start Nginx in the foreground
CMD ["nginx", "-g", "daemon off;"]
\end{lstlisting}

Die Dockerfile ist in einem zweistufigen Build- und Deployment-Prozess aufgebaut.
Der erste Prozess ist die Build-Phase, in der ein Node.js-Image genutzt wird, um die React-App zu bauen. Darüber hinaus werden alle notwendigen Abhängigkeiten installiert und vorallem werden die dynamischen Umgebungsvariablen gesetzt, um spezifische Subbranddaten wie Ids, oder API-URLs zu definieren. Am Ende wird die React-App gebaut und im Ordner dist abgelegt.
Der zweite Prozess ist die Deployment-Phase. Hier wird Nginx als Webserver genutzt und alle generierten Dateien aus der Build-Phase in den Nginx-Container kopiert.
Zuletzt wird der Container bereitgestellt und über Port 80 erreichbar gemacht.
Mithilfe dieser Vorgehensweise werden die einzelnen Subbrands in Container gekapselt und können trotzdem über API-Aufrufe verändert und verwaltet werden.