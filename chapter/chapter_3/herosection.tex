Die HeroSection ist eine Komponente, abgelegt unter „rfinnotrade/subbrand/src/components/HeroSection.jsx“, welche als ein sinnvolles Beispiel einer Komponente im Folgenden beschrieben wird.
In der Datei werden zunächst die notwendigen Daten mithilfe der fetchPageByContentId-Funktion geladen.

\begin{lstlisting}[language=JavaScript, caption={HeroSection.jsx}, label={lst:herosectionjsx}]
useEffect(() => {
    const loadPage = async () => {
        try {
        const data = await fetchPageByContentId();
        setPageData(data);
        setLoading(false);
        } catch (error) {
        console.error("Failed to fetch page", error);
        setLoading(false);
        }
    };
\end{lstlisting}

Dieser typische API-Aufruf kann grundsätzlich in allen Komponenten vorgenommen werden, um gewünschte Daten aus dem Strapi Backend zu laden.
Bis alle Daten geladen sind, wird ein Ladezustand ausgegeben.

\begin{lstlisting}[language=JavaScript, caption={HeroSection.jsx}, label={lst:herosectionjsx}]
return (
    <div className="hero-section">
        {loading ? (
        <h1>Loading...</h1>
        ) : (
        <>
        {pageData?.HeroPicture && <img src={`${import.meta.env.VITE_STRAPI_BASE_URL}${pageData?.HeroPicture?.url}`} alt="HeaderBild" className="hero-image" />}
        </>
        )} 
    </div>
    );
};
\end{lstlisting}

Der Ladezustand wird angezeigt, bis alle Inhalte erfolgreich geladen wurden. Erst dann werden die Inhalte gerendert und angezeigt.