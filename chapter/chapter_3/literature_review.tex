In this chapter, we will describe what, how and why we have selected our literature. 
The significance of the literature review lies not only in its ability to contextualize our study within existing scholarship but also in its multifaceted connections to our research questions, chosen methodologies and eventual findings. We followed the upcoming aspects to investigate our research question. At first we answered the question, whether our work has already been done or not. (Vgl. Adams et al., 2014, S. 34). There are already a lot of articles about what WebAssemby is, what it may can do or especially a performance comparison between WebAssembly and JavaScript. In the great world of software and hardware testing, it has turned out, that slightly differences can change the output a lot. The more tests you make, the more results you have and the more you can compare different results on different bases.
In the next step, we pointed out, what the main theoretical perspective is (Vgl. Adams et al., 2014, S. 34). There are some cases in which WebAssembly is faster than JavaScript and there are also cases, in which JavaScript is faster than WebAssembly. It depends on which explicit test you want to compare JavaScript and WebAssembly (Vgl. De Macedo et al., 2022, S. 10). WebAssembly in general promisis a high and optimised performance (Vgl. WebAssembly – Mehr als nur ein Web-Standard, o. J., S. 1) .On the other hand, JavaScript is the common web development language with all its common issues (Vgl. Common JavaScript Issues and Its Solutions, o. J., S. 1). WebAssembly is build in a binary format, which makes it theoretically faster than Javascript. You can compare these two  on different aspects like Runtime, Memory or cpu usage (Vgl. Sunarto et al., 2023, S. 1).
After that, we want to show up the problems we got while researching the literature (Vgl. Adams et al., 2014, S. 37). While gaining general information about the main theoretical perspectives, the problem was, that there was not that single literature which could answer our research question. It was also not easy to find out, how to make our specific tests between these two architectures. There are many different literatures which compared WebAssembly and JavaScript on many different technical aspects but not pointed it, which method may be the best.
In the following we will describe the major controversies on our topic (Vgl. Adams et al., 2014, S. 37f). WebAssembly is not used in common yet. JavaScript is still the most common web development language. One reason for that can be the disadvantage, that a garbage collector is still missed in WebAssembly, which means that storage management can not be done by itself. WebAssembly can also not interact with the DOM of a Website without JavaScript. (Vgl. Was ist WebAssembly (Wasm)?, o. J., S. 1)  That means, that it cannot operate alone in the web.
