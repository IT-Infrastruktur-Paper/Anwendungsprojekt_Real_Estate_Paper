Since this paper is going to do tests on the interaction of JavaScript and WASM with the DOM (Document Object Model), it is necessary to get a better understanding of the DOM, where it came from and what the benefits and drawbacks are.

The DOM originates from the need to dynamically interact with objects on a website through JavaScript. The standardization and documentation of the DOM was done by World Wide Web Consortium (W3C) as a platform-independent API, accompanied by a lot of stakeholders such as the Document Object Model Working Group and vendors of e.g. HTML or XML editors (!Quelle vgl. w3org). This first officially standardized version by the W3C is called DOM Level 1 and was released in 1998. From that point onward, DOM Level 2 was released in the year 2000 and DOM Level 3 was released in 2004, each with their own subsequent set of enhanced features and capabilities. The last iteration of the DOM, DOM Level 4, was released in the year 2015 (!Quelle vgl. freecodecamp).

Due to its importance as the backbone structure of every modern website that needs dynamic interaction it is important to have a good general understanding of the DOM and its functionality. Therefore ChatGPT has been asked to summarize the available resources on the DOM with the following result: \dq
The Document Object Model (DOM) refers to a standardized, platform-independent application programming interface (API) utilized for representing and interacting with structured documents. Primarily employed in web development, the  DOM provides a hierarchical representation of documents, enabling programmatic access and manipulation of their content, structure, and style. Characterized by its tree-like structure, the DOM organizes elements of an  document into a logical arrangement, wherein each element is represented as a node possessing distinct properties and relationships with other nodes. This model facilitates dynamic manipulation of web content through scripting languages such as JavaScript, allowing for the modification of document elements, attributes, and event handling, thereby facilitating dynamic and responsive web experiences. As a fundamental component of web technologies, the  DOM serves as an intermediary layer between web documents and scripting environments, facilitating seamless integration and manipulation of web content for diverse interactive applications.\dq (!Quelle/Prompt einfügen!)

Of further importance for this paper are the potential bottlenecks developers face when working with the DOM. Performance bottlenecks, stemming from inefficient DOM manipulation and navigation, can diminish the responsiveness and user experience of web applications. Moreover, cross-browser inconsistencies and compatibility are major challenges for developers looking to create consistent experiences across different platforms and devices. Security vulnerabilities, such as cross-site scripting (XSS) attacks and DOM-based injection, underline the importance of implementing robust security measures to safeguard against malicious exploitation. Due to its steep learning curve and the general complexity of DOM manipulation, it is very difficult to learn for developers. (!Quelle vgl. smashingmagazine, keycdn, w3schools)

In conclusion, the HTML Document Object Model (DOM) represents a cornerstone of modern web development, enabling dynamic interactions between web content and scripting environments. Despite its transformative impact and widespread adoption, the DOM is not immune to limitations, which makes it a perfect candidate for the performance tests of this paper.