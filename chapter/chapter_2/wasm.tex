WASM is a standardized bytecode that has been supported by modern browsers since 2017 (see WASM - More than just a web standard, n.d., p. 1). It is intended as a supplement to JavaScript in the browser, as the latter cannot deliver the necessary performance when performance requirements are high (cf. zeroshope, 2020, p. 1).
WASM was declared an official web standard by the World Wide Web Consortium. Apple, Google, Microsoft Mozilla and game engine manufacturers have also participated in the development of the new web standard (see \dq WASM is standard\dq, 2019, p. 49).
In addition, WASM is executed in a so-called virtual machine and there are programming languages such as Rust, C/C++ or Go, which can be translated directly into the WASM bytecode. This means that WASM is not written directly in bytecode, but in one of the aforementioned programming languages and then automatically translated into bytecode. (See WASM - More than just a web standard, n.d., p. 1).
WASM is not a programming language written by humans. WASM is much more a code that is to be written by a machine (see JavaScript vs WASM, n.d., p. 1). This WASM code is the result of compiled, conventional code, for example from a programming language such as Rust or Go. This compiled code is then available in a tightly packaged binary file and can be executed directly by the computer's processor. This directly executable type of code is called low-level code (see What is WASM (Wasm)?, n.d., p. 1).
In order to be able to use web assembly in a browser that supports the technology, the compiled binary file must be executed in this browser using JavaScript. Execution creates a virtual instance within JavaScript, in which the WASM file is then executed (see What is WASM (Wasm)?, n.d., p. 1).
