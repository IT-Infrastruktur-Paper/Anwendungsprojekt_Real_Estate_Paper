An interpreter, in the context of programming, is a software that reads source code line by line and executes the source code directly at this point in time, the so-called runtime. The interpreter therefore analyzes the source code during execution and also only detects potential errors during execution. An example of a programming language that uses an interpreter is Java.
The functionality of an interpreter therefore enables a program to execute the source code directly without having to translate it completely first. Although a translation takes place, this is not separate from the execution, unlike the compiler. For each line in the source code, an immediate action takes place in the form of the execution of the underlying code. The sequence of these actions is determined by the instructions in the source code.
The advantage of an interpreter is that errors that occur during programming can be detected immediately during execution. If there is an error in the source code, the interpreter stops further execution of the program and the programmer recognizes that there must be an error in the source code at precisely this point.
At the same time, the interpreter also has disadvantages. Compared to a compiler, the interpreter works relatively slowly. This is because the interpreter reads each line individually. Even if lines of code are repeated, they are analyzed by the interpreter and only then successfully executed (!Quelle Vgl. Was ist ein Interpreter?, 2019, S. 1).