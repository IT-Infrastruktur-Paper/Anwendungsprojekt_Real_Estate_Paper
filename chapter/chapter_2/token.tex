Die nächste Datei ist unter dem Pfad „rfinnotrade/strapi/src/middlewares/Token.js“ abgelegt.
Diese Datei ist ebenfalls sehr wichtig, da Strapi von Haus aus keine Cookies verarbeiten kann. Die Verarbeitung von Cookies ist in unserem Fall notwendig, damit Informationen wie der aktuelle Login-Status oder welcher User, mit welchen Berechtigungen, startet gerade eine Anfrage an Strapi, abgefragt werden können.
HIER CODE EINFÜGEN
Realisiert wird diese Funktion im ersten Schritt durch das Auslesen des Cookies aus dem mitgelieferten Gesamtkontext der Webseite in Zeile 2 und 3. Im nächsten Schritt wird in den Zeilen 6 bis 8 der JSON Web Token (JWT) extrahiert. Sollte die Abfrage für ein JWT erfolgreich sein, wird in Zeile 9 und 10 das Token um „Bearer“ erweitert und dem „Authorization-Header“ des Webseitenkontextes wieder hinzugefügt.
Durch diese Middleware Funktion versteht Strapi, ob ein User berechtigter Weise eingeloggt ist oder Serveranfragen durchführen darf.