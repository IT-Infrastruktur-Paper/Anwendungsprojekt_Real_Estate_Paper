Rust has emerged to become a very popular programming language in recent years, gathering attention for its innovative approach to addressing common pitfalls in software development, particularly those related to memory safety and concurrency. For the matter of this paper, it is important to take a deeper look into Rust, where it came from, what its main benefits are as well as Rusts problems. 
Rust has quite a unique history, as it was initially developed in 2006 by a young, 29 year old developer named Graydon Hoare in his spare time. It was not until 2010 that the company he worked for, Mozilla Research, presented a project to create a new programming language. This new project was aimed at creating a safe and concurrent alternative to existing systems programming languages. Over the years, Rust has undergone significant development, resulting in its stable release in 2015. (!Quelle vgl. MIT) Hoare took inspiration from existing programming languages such as C++, Haskell, and Erlang. Rust aimed to combine the performance of low-level programming languages with the safety, readability and "ease-of-use" of high-level programming languages. Early versions of Rust focused on refining its type system, borrowing rules, and ownership model, laying the groundwork for its distinctive approach to memory management and concurrency.

Now that the history on Rust is clear, it is more than useful to get a short summary on the functionalities of Rust. Therefore, ChatGPT has been asked to write a short summary of Rusts key features with the following output:
Rust is distinguished by several key features that set it apart from traditional programming languages. Central to Rust's design is its ownership model, which ensures memory safety and prevents common issues such as data races and null pointer dereferences. In Rust, every value has a unique owner, and ownership can be transferred or borrowed through explicit rules enforced by the compiler. This approach eliminates the need for garbage collection while guaranteeing memory safety at compile time. Furthermore, Rust's expressive type system enables developers to write code that is both efficient and easy to reason about. The language supports static typing, generics, traits, and pattern matching, facilitating the creation of robust and scalable software solutions. Additionally, Rust's zero-cost abstractions enable developers to write high-level code without incurring runtime overhead, making it well-suited for performance-critical applications.

With the positive sides of Rust in mind the following section will focus on the drawbacks and limitations of Rust. One of the main hurdles for programmers is the steep learning curve associated with Rust's complex concepts like ownership, lifetimes, and borrowing. Although these features add to the language's safety and efficiency, they can pose difficulties for developers (!Quelle vgl. tutorialspoint, medium)
Another concern related to Rust the compilation time of its projects, which can be significantly longer compared to other languages. This is because Rust does not compile one file at a time but a whole package of files called "Crates", which can take a while to compile (!Quelle vgl. tutorialspoint, medium). While optimizations and tooling improvements have mitigated this issue to some extent, it remains a point of discussions for developers working on time-sensitive projects.
Rust's limited codebase and lack of an elaborate library is another drawback. While efforts have been made to improve compatibility through tools such as the Rust FFI (Foreign Function Interface), seamless integration with legacy code remains an area of active research and development (!Quelle vgl. tutorialspoint, medium).

Overall Rust has become known to be one of the most beloved modern programming language in the past years. It becomes clear that Rust has potential to become even more popular and widely used looking at statements such as 
"It’s enjoyable to write Rust, which is maybe kind of weird to say, but it’s just the language is fantastic. It’s fun. You feel like a magician, and that never happens in other languages." (!Quelle MIT) or the fact that many large companies like Amazon, Dropbox or Microsoft already implemented it in their development process.