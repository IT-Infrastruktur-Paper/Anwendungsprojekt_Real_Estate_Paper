Neben dem eigentlichen Front- und Backend, gibt es weitere Programmiersprachen, die bei der Entwicklung einer Webseite eine Rolle spielen. In diesem Kapitel gehen wir genauer auf diese Sprachen und deren Funktionen ein.

JavaScript ist eine diversifizierte und leistungsfähige Programmiersprache, die in vielen Webentwicklungen eine Rolle spielt. JavaScript wird dabei clientseitig, also lokal auf dem jeweiligen Endgerät des Nutzers, ausgeführt. Dies ermöglicht unter Anderem den Einsatz von clientseitiger Validierung, wodurch Daten bereits lokal überprüft und somit die Serverlast reduziert werden kann\vglfootcite{sikora_professionelles_nodate}.
Der Einsatz von JavaScript ermöglicht außerdem die Implementierung von Logik und Interaktivität, wodurch die dynamischen Benutzeroberflächen im Frontend erst geschaffen und funktional gestaltet werden können. Konkret wird mit JavaScript die Manipulation des DOM ermöglicht\vglfootcite{autor_funktionen_nodate}.
Darüber hinaus können mithilfe von JavaScript Benutzerinteraktionen wie Klicks, Tastatureingaben oder Mausbewegungen verarbeitet werden. Durch den gleichzeitigen Einsatz von AJAX-Technologien können Daten asynchron mit dem Server ausgetauscht werden, wodurch die nahtlose Aktualisierung von Webseiteninhalten  und eine echte und dynamische Interaktivität sichergestellt wird\vglfootcite{autor_javascript_nodate}.
Die grundsätzliche Konzeption von JavaScript sieht vor, dass wiederverwendbare Funktionen als Kernbaustein genutzt und bei Bedarf aufgerufen werden. JavaScript unterstützt verschiedene, gängige Datentypen wie Zahlen, Zeichenketten und Objekte. Dadurch ist es auch möglich, einen objektorientierten Ansatz zu verfolgen.

Auch HTML spielt bei der Entwicklung von modernen Webseiten eine Rolle. Über HTML wird die grundlegende Strukturierung der Benutzeroberfläche vorgenommen.
Anders als bei klassischen Webseiten, wird diese Strukturierung allerdings in JavaScript XML (JSX), eine Syntaxerweiterung für JavaScript, vorgenommen.
JSX ermöglicht es Entwicklern, HTML-artige Elemente direkt in JavaScript umzusetzen, ohne explizite HTML-Syntax zu nutzen\vglfootcite{w3schools_react_nodate}.

Bei Cascading Style Sheets (CSS) handelt es sich um eine deklarative Sprache, die es ermöglicht Webseiten und deren HTML-Elemente visuell zu gestalten. Dabei gibt es verschiedene Ansätze und Möglichkeiten, wie CSS implementiert und verwaltet wird.
Zum einen gibt es die klassische CSS-Variante, bei der eine separate .css-Datei erstellt wird. In dieser Datei werden alle Styles für die Webseite definiert, was bei größeren Projekten allerdings zu Problemen bei der Wartbarkeit führen kann\vglfootcite{autor_welche_2019}.
Eine weitere Möglichkeit bieten CSS Module, die beliebig oft wiederverwendet werden können.
Die dritte Möglichkeit bieten CSS Frameworks. In diesem Fall handelt es sich um vorgefertigte Komponenten und Stile, die ebenfalls wiederverwendet werden können. Auch hier gibt es wieder verschiedene Frameworks wie zum Beispiel Bootstrap, Tailwind CSS oder Foundation.