Betrachten wir als nächstes unser Frontend. Die Hauptkomponente im Frontend ist die Datei App.jsx unter „rfinnotrade/frontend/src/App.jsx“. Diese Datei organisiert die gesamte Struktur der Webseite, steuert die Navigation und verwaltet darüber hinaus auch, in Verbindung mit den Backendkomponenten, den aktuellen Authentifizierungsstatus des Benutzers.
HIER CODE EINFÜGEN, ZEILE 45 bis 63
Der „Router“ Bereich wird verwendet, um die Navigation zwischen verschiedenen Seiten zu ermöglichen. Die Routes innerhalb dieser Sektion definieren die konkret verfügbaren Seiten.
Außerdem ist hier die Sidebar- und Header-Steuerung zu finden. Der aktuelle Zustand der Sidebar wird über „isOpen“ gesteuert und über die „toggleSidebar“ Funktion, kann zwischen den Zuständen der Sidebar gewechselt werden.
HIER CODE EINFÜGEN, ZEILE 20 bis 43
Außerdem wird in Zeile 21 ein state „isAuthenticated“, also ein veränderbarer Zustand, erzeugt und mit false initialisiert. Dazu gehört die checkAuth Funktion, welche eine Anfrage an Strapi ins Backend zur Authentifizierungsüberprüfung sendet. Diese Funktion wird bei jedem Neuladen der Seite automatisch über die ebenfalls implementiert Funktion useEffect aufgerufen.
Sollte der Nutzer abgemeldet werden oder für gewisse Funktionen der Webanwendung nicht freigeschaltet sein, wird über die hier implementierten Funktionen die Authentifizierung stets überprüft und in diesem Fall auf False gesetzt.
Die App.jsx Datei kommt immer beim Laden der Webseite zum Einsatz. Darüber hinaus wird sie genutzt, wenn der Benutzer zwischen verschiedenen Seiten der Webandwendung navigiert, damit die jeweiligen Komponenten auch geladen werden. Damit der Benutzer auch nur auf jene Bereiche und Komponenten zugreifen kann, auf die er auch zugreifen darf, sind die Authentifizierungsfunktionen implementiert.