Für eine funktionsfähige und moderne Webseite ist das Zusammenspiel der zuvor genannten Technologien unerlässlich. Für unser Projekt bilden die klare Trennung von Backend und Frontend sowie die Integration von HTML, CSS und JavaScript, das grundlegende Fundament.

Auf Grund der genannten klaren Trennung von Frontend und Backend, können die Arbeiten an den beiden Teilprojekten unabhängig voneinander und parallel stattfinden. Außerdem bewirkt die Trennung eine gewisse Flexibilität, da Änderungen in einem der beiden Bereiche, nur geringe Auswirkungen auf den anderen Bereich haben. Darüber hinaus können beide Bereiche ebenfalls unabhängig voneinander erweitert oder skaliert werden.
Die Kommunikation zwischen den beiden Bereichen erfolgt über RESTFUL APIs\footcite{Vgl. ehsan_restful_2022}. Diese APIs werden von Strapi im Backend automatisch generiert und von React im Frontend genutzt. Durch diese Vorgehensweise wird eine effiziente Datenverwaltung ermöglicht. Strapi verwaltet die Inhalte und React präsentiert sie. Die in Strapi gespeicherten Inhalte können dabei mehrfach von React im Frontend wiederverwendet werden.
HTML per JSX, CSS und JavaScript müssen im Gesamtkontext eingegliedert werden. HTML wird in React per JSX umgesetzt, um die Arbeit für Entwickler zu vereinfachen und um die grundlegende Webseitenstruktur zu bauen. CSS setzt an dem Punkt an, an dem das Design angepasst und verfeinert wird, um eine anschauliche Benutzeroberfläche zu bauen. JavaScript ist das Bindeglied zwischen den verschiedenen Technologien. Insbesondere in Form von React, wird per JavaScript der Datenfluss und die dynamische Aktualisierung der Benutzeroberfläche logisch umgesetzt und verwaltet.
Docker fungiert hier als Isolation für die ausgeführten Komponenten. Sowohl das Backend als auch das Frontend sollen innerhalb der einzelnen Container laufen.