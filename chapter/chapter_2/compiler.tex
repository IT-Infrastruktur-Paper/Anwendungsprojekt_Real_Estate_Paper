A compiler, in the context of programming, is software that translates the source code completely into a form that can be processed by a machine. The compiler therefore analyzes the complete source code before execution and also detects potential errors in the source code before the first execution. An example of a programming language that uses a compiler is C.
The procedure of the compiler differs not only in the separation of the analysis and execution of the code from the interpreter, but also in the general procedure.
The compiler translates the source code by checking the syntax and then checking the semantics. If the source code passes this process without errors, it is translated into machine code, which can then be executed by the computer.
The advantage of a compiler is that it only has to translate the source code once and can therefore be more performant than an interpreter (Vgl. Was ist ein Compiler?, 2019, S. 1).
