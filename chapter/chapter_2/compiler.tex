Ein Compiler, im Kontext von Programmierung, ist eine Software, die den Quellcode vollständig in eine Form übersetzt, welche von einer Maschine verarbeitet werden kann. Der Compiler analysiert also den vollständigen Quellcode vor der Ausführung und erkennt auch potenzielle Fehler im Quellcode vor der ersten Ausführung. Ein Beispiel für eine Programmiersprache, welche auf einen Compiler zurückgreift, ist C. (Vgl. Was ist ein Compiler?, 2019, S. 1).
Die vorgehensweise des Compilers unterscheidet sich dabei nicht nur in der Trennung der Analyse und der Ausführung des Codes vom Interpreter, sondern auch in der generellen Vorgehensweise.
Die Übersetzung des Quellcodes durch den Compiler erfolgt mit der Überprüfung der Syntax und im Anschluss daran mit der Prüfung der Semantik. Durchläuft der Quellcode diesen Prozess fehlerlos, erfolgt die Übersetzung in den Maschinencode, welcher nun vom Computer ausgeführt werden kann (Vgl. Was ist ein Compiler?, 2019, S. 1).
Der Vorteil eines Compilers liegt darin, dass dieser den Quellcode nur ein einziges Mal übersetzen muss und somit performanter als ein Interpreter sein kann.
