The programming JavaScript is without a doubt the most commonly used programming language on the planet, in fact, about 63\% of developers are using JavaScript according to a 2023 study from Stackoverflow (!Quelle Stackoverflow Survey 2023). This marks JavaScipts 11th year in a row as the most popular programming language. Because of its popularity and importance for this research it is necessary to look further into the history of JavaScript and why it so important for web development.

The programming language JavaScript was initially invented in the year 1995 by Brendan Eich(w3). Two years later it was presented to the ECMA (European Computer Manufacturers Association) to become the ECMA-262 standard in the year 1997 (w3). After the first version (E1), which was solely supported by IE4 (Internet Explorer).
After a lot of iterations, discussions and new major releases, the most important version in the history of JavaScript is version number 5 (E5), which was the first version to be supported by all major browser at the time.

It is also important to declare the strict difference between the two programming languages JavaScript and Java since they have name similarities. Although they where developed in the same year (1995) they are different in its core. The main difference is that JavaScript is a lightweight, dynamically interpreted scripting-language. Java on the other hand is an object-oriented programming language which is made to be more robust and platform independent, leading with the promise "Write Once, Run Anywhere".

In order grasp the complexity and importance of JavaScript, it is necessary to get a basic understanding of it. Therefore ChatGPT has been asked to summarize the available resources on JavaScript with the following result:

"JavaScript is a high-level, dynamic, interpreted programming language primarily utilized for client-side web development. It employs a prototype-based object-oriented paradigm, featuring first-class functions and asynchronous programming capabilities. JavaScript is commonly embedded within HTML documents and executed by web browsers to manipulate the Document Object Model (DOM) dynamically, enabling interactive user interfaces and dynamic content generation. Its versatility extends beyond web development, as it can also be employed for server-side scripting, mobile application development, and desktop application development through frameworks like Node.js. JavaScript's syntax is influenced by C and Java, offering a rich ecosystem of libraries, frameworks, and tools for application development. It serves as a fundamental technology underpinning modern web applications, facilitating the creation of dynamic and interactive experiences across various platforms." (!Quelle/Prompt einfügen!)

The main takeaway from ChatGPT's summary is that the programming language JavaScript is of utmost importance for the development and usage of modern websites. Because without JavaScript, many parts of a website would not be dynamic as we have gotten used to it by now but rather static in its code and interaction (!Quelle as cited in freecodecamp). After all, without JavaScript "all you would have on the web would be HTML and CSS" (!Quelle freecodecamp) and commonly known website design-clues like a full-page drop down menu or content that is dynamically loaded into the websites body would be missing without the existence of JavaScript.

After those insights it is further necessary to take a look at common problems and setbacks of JavaScript. One main area of mistakes in JavaScript code is the syntax and data type handling, especially the lack of strong typing and type coercion. According to ChatGPT, the lack of strong typing in JavaScript "can make it prone to runtime errors and debugging difficulties, particularly in larger codebases where type safety is crucial" (!Quelle ChatGPT Prompt Error). The other major source for errors in JavaScript is type coercion, which is a process where a value is converted from one data type to another. The process in itself is not the problem, it is rather that JavaScripts loose typing and its therewith connected automatic type conversion can lead to errors, especially when using the equal operator.
Another major source of errors is nesting multiple callbacks into a pyramid-like structure, which has become known as Callback Hell (!Quelle vgl. ChatGPT Prompt Error \& GeeksforGeeks). Callback Hell usually does not lead to technical errors but rather logical errors since the structure can become difficult to read and to understand. 
The last major error source is JavaScripts bottleneck in performance, caused through "Inefficient algorithms, excessive DOM manipulation, or poorly optimized code" (!Quelle ChatGPT Prompt Error). Especially the excessiveness in which JavaScript allows you to manipulate the DOM seems to be a problem, since it does not provide any guidance on how to do DOM manipulation efficiently (!Quelle vgl. ChatGPT Prompt Error \& Toptal \& Browserstack). Combined with the common problem of memory leaks caused by "retaining references to objects that are no longer needed" (!Quelle vgl. ChatGPT Prompt Error).

Even though it seems like there are a lot of problems and error sources with JavaScript, one needs to keep in mind that JavaScript is in fact the most used programming language in the world, hence a lot of developers have used JavaScript to its bones and found every single problem there is. Overall it is still a very practical, lightweight and easy to learn language which continuously delivers good results.