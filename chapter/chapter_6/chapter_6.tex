\newpage
\section{Conclusion} \label{conclusion}
In conclusion, the findings of this paper provides evidence for the performance advantages of WASM over JavaScript, particularly when executing computationally intensive algorithms. Through a series of tests and respective analyses, we have confirmed that WASM offers notable improvements in CPU utilization, memory efficiency, and overall performance compared to JavaScript.

Two out of the three null hypothesis and one alternate hypothesis (which thought of WASM having a faster runtime) were confirmed, indicating that WASM runtime indeed outperforms JavaScript runtime across various performance metrics. This validation underscores the inherent technological advantages of WASM, including its low-level nature and compilation-based execution, which contribute to its superior performance characteristics. Furthermore, the unexpected result observed in the third test, where WASM showed faster DOM manipulation compared to JavaScript, highlights the complexity of performance evaluation in real-world scenarios. Despite WASM relying on JavaScript for DOM manipulation tasks, its inherent efficiencies shine through, suggesting subtle influences on performance outcomes. What determines those influence is matter to future research.

Our discussion of the underlying mechanisms driving these performance differences, including CPU utilization patterns, memory consumption, and DOM manipulation efficiency, has provided valuable insights into the technical nuances of both WASM and JavaScript platforms. These insights serve to inform developers and decision-makers when selecting the appropriate technology stack for specific use cases, enabling more informed choices and optimized performance outcomes. Overall, this study contributes to the growing body of research on Web Assembly and JavaScript performance, shedding light on their comparative advantages and limitations. By demonstrating the clear benefits of WASM in certain scenarios, we underscore its significance as an advanced technology that enables more efficient execution of code on web browsers, paving the way for enhanced web application performance and user experiences.

As the landscape of web development continues to evolve, further exploration and refinement of Web Assembly's capabilities will be essential, offering continued opportunities for innovation and optimization in the field of client-side application development.