Als nächstes gehen wir auf unsere verwendete Backendstruktur ein. Die von uns gewählte Backend Technologie, nennt sich Strapi.
Strapi ist ein Headless Content Management System (CMS), das Open-Source betrieben wird. Strapi zeichnet sich durch seine Flexibilität und durch seine Entwicklerfreundlichkeit aus. Die Idee hinter Strapi ist die Trennung der Inhaltsverwaltung vom Frontend\footcite{autor_strapi_2024}.
Es ermöglicht eine effiziente Erstellung, Verwaltung und Bereitstellung der Inhalte über eine integrierte und benutzerfreundliche Oberfläche. Der Headless-CMS-Ansatz bedeutet, dass alle Inhalte von der Präsentationsschicht getrennt sind. Alle Daten die in Strapi abgelegt und verwaltet werden, werden per API an das Frontend bereitgestellt\footcite{autor_headless-cms_nodate}.
Durch die Bereitstellung der Daten per API, ist Strapi in der Wahl des Frontends flexibel, da APIs grundsätzlich einem standardisierten Format folgen und jeweils leicht angepasst werden können. 
Im Folgenden werde ich etwas konkreter auf die einzelnen Vorteile von Strapi eingehen.
Ein wichtiger Aspekt von Strapi ist die vorhandene Flexibilität. Strapi ermöglicht es, Inhalte und Strukturen präzise an die eigenen Projektanforderungen anzupassen. Dabei erlaubt es die No-Code-Konfiguration, das Backend nahezu ohne eigene Programmierung einzurichten. Durch die mitgelieferte, grafische Benutzeroberfläche, können sowohl Datenfelder und Inhaltstypen als auch Relationen zwischen den Daten bequem erstellt werden\footcite{autor_headless-cms_nodate}.
In diesem Zusammenhang ist die einfache Integration von Strapi in das Gesamtprojekt zu erwähnen. Es können unterschiedlichste Datenbanksysteme mit Strapi genutzt werden. Darunter SQL-basierte, aber auch NoSQL-basierte. Die Daten wiederum können ganz einfach per API an das unabhängige Frontend übermittelt werden\footcite{autor_strapi_nodate}.
Ein weiterer äußerst wichtiger Aspekt, der von Strapi abgedeckt wird, ist die IT-Sicherheit. Strapi verfügt bereits von Haus aus, über grundlegende Sicherheitsfunktionen wie Authentifizierung und Authorisierung. Zudem liefert Strapi die Möglichkeit zur Konfiguration von Rollen- und Berechtigungskonzepten direkt mit\footcite{autor_basics_nodate}.