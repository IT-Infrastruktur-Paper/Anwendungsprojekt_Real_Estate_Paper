Das Projekt besteht aus mehreren wichtigen Komponenten, die eng miteinander verzahnt sind:

\textbf{GitHub Repository rfinnotrade}

Das Herzstück des Projekts ist das GitHub Repository rfinnotrade. Hier befindet sich die gesamte Projektstruktur mit dem zugehörigen Code. Das Repository gliedert sich in drei zentrale Bereiche. Im Frontend-Bereich ist die gesamte Benutzeroberfläche sowie alle clientseitigen Komponenten implementiert. Das Backend, basierend auf Strapi, verwaltet die Datenbank und stellt die notwendigen REST-APIs bereit. Der Subbrands-Bereich enthält sämtliche Templates und Konfigurationen für die Webseiten, die über das Admin Center erstellt werden. Die Entwicklung erfolgte in Visual Studio Code mit Remote-SSH-Zugriff auf das Repository, was eine sichere und effiziente Zusammenarbeit ermöglichte.

\textbf{GitHub Repository WebPaper}

Für die Dokumentation wurde ein separates LaTeX-Repository eingerichtet. Die Entscheidung für LaTeX als Dokumentationswerkzeug basierte auf mehreren Vorteilen. Zunächst ermöglicht es eine erhebliche Kosteneinsparung im Vergleich zu kommerziellen Lösungen wie Overleaf. Durch die Git-Integration wird eine präzise Versionskontrolle gewährleistet. Die Projektgruppe konnte außerdem auf eine bewährte Projektstruktur aus vorherigen Arbeiten zurückgreifen. Nicht zuletzt überzeugt LaTeX durch das professionelle Erscheinungsbild der generierten PDF-Dokumente. Die gesamte Dokumentation konnte so effizient im Team erstellt und verwaltet werden.

\textbf{Server}

Der Server spielt eine duale Rolle im Projekt. In seiner Funktion als Produktivumgebung ist er für das Hosting der Live-Webseiten und Dienste zuständig. Als Entwicklungsumgebung ermöglicht er die zentrale Verwaltung verschiedener Aspekte: Die Entwicklungsdependenzen werden hier einheitlich verwaltet, alle notwendigen Umgebungsvariablen sind zentral konfiguriert, Build-Prozesse werden standardisiert durchgeführt und Deployment-Workflows sind klar definiert. Diese Zentralisierung hat die Teamarbeit deutlich vereinfacht und Inkonsistenzen zwischen verschiedenen Entwicklungsumgebungen verhindert.

\textbf{Figma}

Figma diente als kollaborative Design-Plattform für verschiedene Projektbereiche. Für die Hauptwebseite www.rf-innotrade.de wurde ein responsives Layout entwickelt, das sich an verschiedene Bildschirmgrößen anpasst. Das Admin Interface wurde als benutzerfreundliche Verwaltungsoberfläche mit intuitiver Navigation konzipiert. Für den Authentifizierungsbereich wurde ein sicheres und modernes Login-Design erstellt. Das Kontaktformular wurde für optimale Benutzerfreundlichkeit bei Kundenanfragen gestaltet. Besonders wichtig war die Entwicklung eines flexiblen Design-Systems für die automatisch generierten Unterseiten. Durch die Echtzeitkollaboration in Figma konnte das Team Design-Entscheidungen schnell treffen und iterativ verbessern.