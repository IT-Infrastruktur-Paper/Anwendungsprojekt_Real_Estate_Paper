\newpage

\section{Einleitung} \label{einleitung}

Die Entwicklung des Immobilienmarktes steht seit Jahren im Fokus von Politik, Wirtschaft und Gesellschaft. Insbesondere die Entwicklung der Immobilienpreise hat weitreichende Auswirkungen auf die Volkswirtschaft, die Sozialstruktur und die individuelle Vermögensbildung. Eine präzise Analyse und Vorhersage von Kaufpreisen für Immobilien ist daher nicht nur für Investoren und Banken, sondern auch für politische Entscheidungsträger und Privatpersonen von großer Bedeutung.
Der Immobilienmarkt ist durch eine hohe Dynamik und Komplexität gekennzeichnet, da zahlreiche Faktoren die Preisbildung beeinflussen. In Städten herrscht oft eine andere Dynamik als in ländlicheren Regionen, weshalb die genaue Kenntnis der Einflussfaktoren und deren Wechselwirkungen eine zentrale Voraussetzung für fundierte Marktanalysen ist. Zudem gewinnt der Einsatz von datenbasierten Methoden in der Immobilienforschung stetig an Bedeutung, da sie es ermöglichen, große und heterogene Datenmengen systematisch auszuwerten und bislang verborgene Muster zu erkennen. Moderne Analyseverfahren, wie maschinelles Lernen (ML), bieten dabei neue Möglichkeiten, komplexe Zusammenhänge zu modellieren und Prognosen auf einer breiteren Datenbasis zu erstellen.
Frühere Untersuchungen zur Prognose von Immobilienpreisen stützen sich überwiegend auf ökonomische Kennzahlen und objektbezogene Merkmale, wie z.B. das Zinsniveau oder den Zustand und die Wohnfläche der Immobilien. Die Nutzung freizugänglicher regionalstatistischer Daten, wie sie durch den Deutschlandatlas zur Verfügung gestellt werden, wurde bislang kaum untersucht. Der Deutschlandatlas ist ein umfassendes Datenportal mit regionalen Strukturdaten, dass eine Vielzahl von Indikatoren enthält, die potenziell als Grundlage für die Ermittlung und Vorhersage von Immobilienkaufpreisen dienen können.

Die zentrale Forschungsfrage der vorliegenden Arbeit lautet: 
Wie geeignet ist der Deutschlandatlas als Grundlage zur Ermittlung und Vorhersage von Immobilienkaufpreisen in Deutschland?

Zur Beantwortung der Forschungsfrage wird eine Datenanalyse mit Python durchgeführt. Zunächst werden statistische Zusammenhänge zwischen den Indikatoren des Deutschlandatlas und den Immobilienpreisen untersucht. Anschließend wird ein einfaches Prognosemodell implementiert, um die wichtigsten Einflussfaktoren datenbasiert zu identifizieren und die Entwicklung der Immobilienpreise vorherzusagen. Im folgenden Kapitel werden zentrale Grundlagen für die weitere Analyse gelegt. Im anschließenden dritten Kapitel wird das methodische Vorgehen anhand des CRISP-DM-Modells dargestellt. Das vierte Kapitel umfasst die Datenaufbereitung, die Modellierung und Auswertung mit Python. Abschließend werden die Ergebnisse dargestellt und diskutiert, sowie die zentralen Erkenntnisse zusammengefasst.