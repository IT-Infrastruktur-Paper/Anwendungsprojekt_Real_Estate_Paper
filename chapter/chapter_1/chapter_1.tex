\newpage

\section{Introduction} \label{introduction}
The evolution of web technology has continuously pushed the boundaries of what can be achieved within web applications, particularly in terms of performance and computational capabilities. One of the latest advancements in this arena is Web Assembly (WASM), a binary instruction format designed to enable high-performance code execution on web browsers. As the demand for sophisticated web applications increases, it is imperative to understand the comparative advantages of WebAssembly (WASM) over traditional JavaScript (JS), particularly in scenarios involving computationally intensive operations (!Quelle vgl. SitePoint &  WebAssembly Concepts).

JavaScript has been the predominant language for web development for the last decades. However, the interpreted nature and runtime environment of JavaScript introduce performance bottlenecks, especially when dealing with complex computations. Therefore, it is important to consider the benefits of using WebAssembly for implementing client-side logic. WebAssembly offers a promising alternative by providing a low-level bytecode format that can be efficiently compiled and executed across different platforms (!Quelle vgl. SitePoint &  WebAssembly Concepts).

The aim of this thesis is to investigate the performance differences between Web Assembly and JavaScript in client-side applications, with a specific focus on computationally intensive operations. Through rigorous experimentation and analysis, we aim to determine whether Web Assembly provides a distinct advantage over JavaScript in terms of efficiency.

To achieve this aim, the research will investigate various aspects of performance comparison, such as CPU utilization, memory usage, and DOM manipulation. We will conduct benchmark tests and empirical evaluations to provide concrete insights into the relative strengths and weaknesses of both approaches.

Furthermore, this research will not only clarify the technical aspects of Web Assembly and JavaScript but also investigate their practical implications for developers aiming to enhance performance in real-world situations. By explaining the trade-offs, limitations, and best practices of each technology, we aim to provide useful guidance for making informed decisions when choosing the most appropriate technology for client-side applications that require computationally intensive operations.

\subsection{Research Aim} \label{aim}
This paper aims to investigate the performance disparity between Web Assembly (WASM) and JavaScript (JS) in the context of client-side applications, with a focus on computationally intensive algorithms. The objective is to determine whether WASM provides a distinct advantage over JS in terms of efficiency through comprehensive experimentation and analysis.
\subsecttion{Research Objectives} \label{objectives}
In order to completely cover the most important topics and to have a clear outline for the research process, we defined the following research objectives:

\begin{itemize}
    \item Evaluate the performance of Web Assembly and JavaScript implementations in executing computationally intensive algorithms within client-side applications.
    \item Measure and compare CPU utilization between WASM and JavaScript implementations across various benchmark tests.
    \item Assess memory usage and management efficiency between WASM and JavaScript for tasks involving computationally intensive operations.
    \item Analyze the impact of DOM manipulation on the overall performance of WASM and JavaScript implementations within the context of client-side applications.
    \item Provide empirical evidence and insights to ascertain whether Web Assembly is advantageous over JavaScript from an efficiency perspective in the targeted application domain.
    \item Offer recommendations and guidelines for developers based on the findings to optimize performance when selecting between Web Assembly and JavaScript for client-side applications with computationally intensive algorithms.
\end{itemize}