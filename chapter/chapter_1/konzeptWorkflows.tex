RF InnoTrade stellt eine Dachfirma im E-Commerce-Bereich dar, die sich durch eine starke Diversifizierung Ihrer Produkte auszeichnet. Der primäre Fokus des Unternehmens liegt darauf, schnell und flexibel auf neue Trends sowie Produkte zu reagieren, die dem Reselling-Markt zur Verfügung gestellt werden. Diese Agilität ist wichtig, um schnellstmöglich auf einen sehr dynamischen Markt reagieren zu können.

Um die technische Umsetzung dieses Ansatzes effizient zu gestalten, ohne bei jeder neuen Marke oder Produkt eine komplett neue Webseite erstellen zu müssen, war eine Lösung erforderlich, die sowohl den Programmieraufwand als auch die damit verbundenen Kosten so gering hält wie möglich. Beispielsweise entfallen typische Aufwendungen für die Registrierung neuer Domains, das Hosting sowie teure Upgrades für Subsysteme wie Shopify (auf dessen Plattform ein Ähnliches, aber sehr kostenintensives, Resultat erzielt werden könnte). Die Lösung sollte einen einfachen, schnellen und kostengünstigen Prozess bieten, der es dem Unternehmen ermöglicht, neue Webseiten für Marken oder Produkte mit minimalem Aufwand zu generieren.

In diesem Kontext wurde die Webseite www.rf-innotrade.de um ein neues Admin Center erweitert. Dieses Admin Center ermöglicht eine effiziente Verwaltung und Erstellung neuer Webseiten mit einem vereinfachten Konfigurationsprozess. Es enthält wesentliche Funktionen, die zur Automatisierung und Skalierung der operativen Abläufe beitragen:

\begin{enumerate}
    \item Die Möglichkeit, mit nur einem Klick eine neue Marke oder Webseite zu erstellen, inklusive eines vollständigen Konfigurationsprozesses.
    \item Eine KI-unterstützte Generierung von Beschreibungsbildern und Texten, die den Content-Erstellungsprozess beschleunigt.
    \item Eine umfassende Übersicht aller erstellten Marken und Webseiten zur effizienten Verwaltung.
    \item Die Möglichkeit, einzelne Marken oder Webseiten on- und offline zu schalten.
    \item Die Option, bestehende Marken oder Webseiten hinsichtlich ihrer Rahmenparameter zu bearbeiten.
\end{enumerate}

Technisch gesehen wird für jede neue Webseite oder Marke ein separater Docker-Container gestartet, der ein vorgefertigtes Image enthält. Vor dem Build-Prozess werden spezifische Variablen zugewiesen, die den Inhalt der jeweiligen Webseite definieren. Diese Variablen sowie der zugehörige Inhalt werden zentral im Content-Management-System Strapi gespeichert, was eine effiziente Verwaltung und Anpassung der Daten ermöglicht.

Das übergeordnete Ziel dieser Erweiterung ist es, RF InnoTrade sowohl organisatorisch als auch technisch in die Lage zu versetzen, mit hoher Geschwindigkeit und Flexibilität zu skalieren. Die Lösung trägt somit zur Optimierung von Prozessen bei und unterstützt das Unternehmen dabei, sich schnell an wechselnde Marktanforderungen anzupassen.

Um ein besseres Verständnis für die Funktionalität des Admin Centers zu erhalten wurden drei beispielhafte Workflows erstellt. Diese Workflows zeigen, wie das Admin Center in der Praxis genutzt werden kann, um neue Marken oder Webseiten zu erstellen, zu verwalten und zu bearbeiten. Die Workflows sind in den folgenden Abschnitten detailliert beschrieben.